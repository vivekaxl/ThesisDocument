\section{Towards Efficient Resource Recommendation}
\label{sec:future_work}

This section presents our proposed research in search for
cost-effective cloud configurations.

\subsection{Configuration Space}
IaaS allows on to choose compute, storage and networking types
at different costs.
We primarily focus Amazon Web Service (AWS)~\cite{AmazonEC2}, but
our approach should be able to apply to other cloud platforms such as
Microsoft Azure and Google Cloud Platform.
Since there are many combinations and
the initial focus is
the \emph{scale-up} and \emph{scale-out} scenarios.

In the \emph{scale-up} scenario, users can choose a configuration from
the instance family such as \emph{m4} and \emph{c4},
and
the instance size such as \emph{large} and \emph{2xlarge}.
\mytable{\ref{tab:ec2_types}} lists parts of supported virtual machines on AWS.
There is a wide range of resource choices as well as costs.

\begin{table}[t]
\centering
\resizebox{\textwidth}{!}{\begin{tabular}{|l|r|r|r|c|r|}
\hline
Instance Type & \$/Hour& vCPU & Memory (GB) & Networking Performance & GHz               \\ \hline
t2.large      & 0.094  & 2    & 8            & Low to Moderate        & Up to 3.0         \\ \hline
t2.2xlarge    & 0.188  & 8    & 32           & Moderate               & Up to 3.0         \\ \hline
m4.large      & 0.108  & 2    & 8            & Moderate               & 2.4               \\ \hline
m4.16xlarge   & 3.447  & 64   & 256          & 20 Gigabit             & 2.3               \\ \hline
c4.large      & 0.100    & 2    & 3.75         & Moderate               & 2.9               \\ \hline
c4.8xlarge    & 1.591  & 36   & 60           & 10 Gigabit             & 2.9               \\ \hline
x1.16large    & 6.669  & 64   & 976          & 10 Gigabit             & 2.3               \\ \hline
x1.32xlarge   & 13.338 & 128  & 1,952        & 20 Gigabit             & 2.3               \\ \hline
r4.large      & 0.133  & 2    & 15.25        & Up to 10 Gigabit       & 2.3               \\ \hline
r4.16xlarge   & 4.256  & 64   & 488          & 20 Gigabit             & 2.3               \\ \hline
i3.large      & 0.156  & 2    & 15.25        & Up to 10 Gigabit       & 2.3               \\ \hline
i3.16large    & 4.992  & 64   & 488          & 20 Gigabit             & 2.3               \\ \hline
\end{tabular}}
\caption{On-Demand EC2 Instance Types}
\label{tab:ec2_types}
\end{table}


The \emph{scale-out} capability is critical for supporting large-scale problems.
Applications are able to complete in a shorter time by using
more machines.
For example, the \emph{MapReduce} programming model~\cite{DeanJ2004_MapReduce}
is highly scalable
because the I/O operations per node remains constant
as the number of machines increases.
A larger cluster size usually means shorter execution time but
comes with a higher resource cost.


\subsection{Low-Level Performance Metrics}

Low-level performance metrics are good indicators for
analyzing application performance.
There are many possible metrics can be used for application characterization.
We choose \emph{β} and \emph{MPO} to understand whether an application
is CPU bound or memory bound~~\cite{Freeh2007}.
These two metrics are particularly useful in the scale-up scenario
because adding more memory is more beneficial to a memory bound application
but not CPU bound one.

Low-level performance metrics such as
CPU utilization, disk I/O, network throughput
are general metrics and easy to collect from operating systems.
In our previous work, we have shown that such low-level metrics
can be used to estimate end-to-end performance of
distributed storage systems~\cite{Hsu2016}.
By leveraging our proposed modeling approach, we are able to predict
application performance with different resource configurations.



\subsection{Search Algorithms}

Since the configuration space is very large and
exhaustive search is not feasible,
an efficient search algorithms that eliminate the number of evaluations
is central to configuration recommendation.
We introduce popular search methods for finding the best configurations.

\paragraph*{Random Search}
This method randomly pick configuration to evaluate.
However, this simplest method is not ideal because
it does not guarantee the outcome.
Furthermore, evaluation cost is very expensive.
Therefore, we will need a more sophisticated and efficient search method.

\paragraph*{Coordinate Descent}
The coordinate descent is a simple search algorithm that
finds the best point along one dimension.
A significant drawback is it can get stuck at a non-stationary point
that yields a local maximum or minimum.
When there are multiple dimensions, as in our case, this search algorithm
might become insufficient~\cite{Alipourfard2017}.


\paragraph*{Evolutionary Algorithm}
This algorithm is inspired by biological processes such as
reproduction, mutation, recombination, and selection.
Evolutionary algorithm is generally performs well because
it does not use any assumption for the underlying fitting landscape
~\cite{Bergstra2011}.
However, evolutionary algorithm requires large evaluations for effective search.
Therefore, this method might not be appropriate for our scenario.


\paragraph*{Bayesian Optimization}
Bayesian optimization has been proved to be an efficient framework
of searching the best parameters and configurations
that are expensive to evaluate
~\cite{Klein2016, Alipourfard2017}.
Bayesian optimization is an instance of sequential model-based optimization,
and  often comes with the Gaussian Process
as a surrogate for modeling the objective function
(\emph{cost} and \emph{time} in our problems).
Acquisition function defines the balance between
exploring new configurations for avoiding local optimal
and
exploiting points for faster converge.



\subsection{Guided Search with Expert Knowledge}

Previous search algorithms focus solely on the configuration space.
We argue that the search algorithms can be augmented
by leveraging low-level system performance metrics.
Low-level performance metrics
such as CPU utilization and page faults are good indicators
to measuring the bottleneck resources.
For example, when we see an application is memory bound and
we also find the application spend large portion of time on memory access,
a configuration with large memory size may have a higher chance to improve
application performance.
Therefore, we can use low-level performance metrics
to distinguish bottleneck resources (limiting factors)
and
unsaturated resources (do not need to scale up).

In a real world application, domain experts collect
a great number of performance metrics for understanding application behavior.
We argue that this process can be automated and thereby can be integrated
to existing search algorithms.
We plan to leverage Bayesian optimization and expert knowledge
for finding out the most cost-effective configurations
in a more efficient way.


\iffalse
  - Our approach leverage Bayesian Optimization and expert's knowledge
  - Bayesian optimization to reduce evaluation cost {[}CherryPick{]}
  - Expert's strategy (low-level performance metrcis) {[}FABOLAS,
  Inside-Out{]}
  - (Saturated resources) Limiting factors needs to be identified
  - (Unsaturated resources) are not that important
  - CDP inforation for knowing the trade-off (exploration or exploitation)
\fi


\iffalse
\begin{itemize}
\item
  Show the wide range of AWS configurations
\end{itemize}

\paragraph{3.2 Low-Level Performance
Metrics}\label{low-level-performance-metrics}

\begin{itemize}
\item
  β: CPU bound
\item
  MPO: memory bound
\item
  Inside-Out shows low-level performance metrics are a good proxy for
  measuring end-to-end performance of a distributed storage systems
\end{itemize}

\paragraph{3.3 Data-Driven Modeling
Approach}\label{data-driven-modeling-approach}

\begin{itemize}
\item
  Regression-based Approach
\item
  Classification Approach
\item
  Behavior pattern
\item
  Anamoly detection
\item
  Deep Learning
\end{itemize}

\paragraph{3.4 Search Optimization}\label{search-optimization}

\begin{itemize}
\item
  Random Search
\item
  Coordinate Descent
\item
  May face local minimum or maximum
\item
  Evolutionary Algorithm (EA)
\item
  Bayesian Optimization (BO)
\end{itemize}

\paragraph{3.5 Guided Search with Expert
Knowledge}\label{guided-search-with-expert-knowledge}

\begin{itemize}
\item
  Previous search algorithms focus on hyperparameters or solely
  configurations. We argue that the search algorithms can be augmented
  by leveing low-level system performance metrics.
\item
  Our approach leverage Bayesian Optimization and expert's knowledge
\item
  Bayesian optimization to reduce evaluation cost {[}CherryPick{]}
\item
  Expert's strategy (low-level performance metrcis) {[}FABOLAS,
  Inside-Out{]}

  \begin{itemize}
  \item
    (Saturated resources) Limiting factors needs to be identified
  \item
    (Unsaturated resources) are not that important
  \item
    CDP inforation for knowing the trade-off (exploration or
    exploitation)
  \end{itemize}
\end{itemize}
\fi