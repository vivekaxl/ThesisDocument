\section{Introduction}
\label{sec:introduction}


This chapter presents \scout, which 
identifies search regions that contain suitable architecture configurations.
\scout follows sequential model-based optimization (SMBO)
that converges to the best architectural configurations.
This method better tolerates high variance of the prediction error
in a machine learning based prediction model.
Second, \scout adopts pairwise comparison for determining
the next architectural configurations
that are likely to improve upon the current choice.
Instead of predicting workload performance directly
(\eg in \emph{CherryPick} and \emph{PARIS}),
\scout uses relaxed modeling to find
the next \emph{better} choices (relative ordering).
This naturally fits into SMBO.
Third, \scout uses low-level performance metrics
to identify performance bottlenecks and
to capture cost-performance relationship.
Last, \scout uses transfer learning to acquire knowledge of CAT
from other workloads.
These four elements enable \scout to navigate through the search space
more quickly and intelligently.

Any search-based method has two aspects.
\vspace{-0.5em}
\begin{itemize}
    \setlength\itemsep{-0.4em}
    \item \textit{Exploration:} Gather more information about the search space by executing a new cloud configuration.
    \item \textit{Exploitation:} Choose the most promising configuration based on information collected.
\end{itemize}
\vspace{-0.5em}
Additional exploration incurs higher search cost and insufficient exploration may lead to sub-optimal solutions.
This is the exploration-exploitation dilemma the appears in many machine learning problems~\cite{kaelbling1996reinforcement}.
For example, \emph{CherryPick} requires a good exploration strategy
to characterize the search space~\cite{Alipourfard2017}.

In this chapter, we demonstrate that it is possible to
trade exploration for exploitation
without settling for a sub-optimal configuration.
The central insight of this work is that the cost of the search
for the right cloud configuration can be significantly reduced
by using information gathered during tuning.
However, prior work such as \emph{CherryPick} and \emph{Arrow}
learns from an optimization process of a single workload
~\cite{Alipourfard2017, Hsu2018Arrow}.
This produces unnecessary search cost on exploration
(related to the \emph{cold-start} problem) and
may eventually lead to sub-optimal choices (due to irregular search space).
\scout is able to alleviate these problems
with \emph{transfer learning}~\cite{pan2010survey}---
the knowledge is transferred from previous (but distinct) workloads
using relative ordering and low-level performance metrics,
which does not require workload information.

In this chapter, we identify the key components for an effective method
to the CAT problem.
\scout enables practitioners to find a near-optimal
cloud architecture configuration
with a better search performance and a lower search cost than
the state of the art.
Our key contributions are:

\begin{enumerate}[leftmargin=*]
\setlength\itemsep{-0.4em}
\item we propose a novel system, \scout, that
finds (near) optimal solutions and solves the shortcomings of prior work.
(Section~\ref{sec:approach});
\item we present a novel way to represent the search space,
which can then be used to transfer knowledge
from historical measurements(Section~\ref{sec:approach});
\item we compare the performance of \scout with
other state-of-the-art methods
using 125 workloads and 87 architecture configurations
on three different data processing systems.
(Section~\ref{sec:evaluation}); 
and
\end{enumerate}