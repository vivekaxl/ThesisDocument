\section{The Inside-Out Design}
\label{sec:methodology}
%\rp{Chin: In the previous section, you say ``As we will explain in Section V-A, we use 32 low-level performance 
%metrics collected with dstat''. This section does not say anything about 32 metrics. Please mention them here. 
%I think we need to write this section in a much better way. You should clearly explain the steps: something like: We collect 32 
%low level system metrics from each VM (either give the list or mention just the few) every x seconds. Then say we take mean, std, sum and 5\% of each 
%of these 32 metrics for every moving window of xxx seconds. Then explain why we use moving window. then say that for training and validation purposes, 
%we measure the end-to-end performance metrics (IOPS, throughput, and latency) using cosbench every x seconds. Then we take average, xxx, xxx etc. 
%of these metrics ...I think you get the point now, right? We have to clearly explain what we exactly did. }
In this section, we present the design of Inside-Out.
We also discuss the trade-offs among a set of representative machine learning algorithms
and propose a two-step learning technique for mitigating overfitting problems.



\subsection{Collecting and Pre-Processing Low-Level Metrics}
\label{sec:data_preprocessing}

Inside-Out collects general, low-level system metrics from individual machines running the distributed storage service.
However, the raw collected data suffers from various problems due to inefficiency of data collectors, system clock skews,
incomparable data formats, workload outliers, bursty system anomalies, \etc
The noisy data can lead to unstable and 
inaccurate performance models. 
Inside-Out performs a series of data pre-processing functions to address these issues.

%\rp{The goal of this step is to remove unwanted characteristics of the collected data, such as
%workload outliers, system clock deviation, incomparable data format, etc.
%}


%Data preprocessing is a preparation step before training a model.


\subsubsection{Monitoring storage components}
We collect low-level system metrics of the underlying operating system to capture resource utilization 
(\eg cpu, memory, disk and network usage).  
The low-level performance metrics are sampled with one-second granularity.
Such data can be collected from \textit{libvirt}, Ganglia, instrumented hypervisors \cite{Koh2007} and \textit{Ceilometer} in OpenStack.
We use  \emph{dstat} monitoring tool (with option: -tcly -mg --vm -dr -n --tcp --float) to collect these data.


%This step helps us eliminate unwanted characteristics of the collected data, such as
%high variation due to workloads, uncertainty originated by physical hardware, 
%system clock deviation, incomparable data format, etc.
%For the collected performance data, fluctuating metrics, time synchronization and feature transformation for a distributed system are the major challenges.

%\subsubsection{Smoothing monitored data}
\subsubsection{Data smoothing}
%
Building a performance model with data collected at one-second granularity is challenging because 
system data can exhibit high variance at small time scales, \eg due to dynamic/bursty workloads and interference among co-located tenants.
Furthermore, the storage IO operation needs to pass through a series of software layers 
between the storage client and the back-end raw physical storage device. 
The long storage IO path can introduce high variability in resource utilization at smaller time scales. 
For example, HDFS and Ceph both replicate data blocks across storage nodes distributed in physically disjoint 
servers, racks or even datacenters.
To address the uncertainties due to complex IO path spanning several software layers, 
we compute the moving average of the collected performance data.
We have empirically found that one-minute window for processing the moving average is sufficient to 
eliminate outliers from the raw data.
%we use moving averages to construct a stable prediction model.
%We choose one-minute windows for moving average since we have empirically found 60-second moving averages to smooth out raw data sufficiently, without 
%compromising data quality, in order to yield stable and accurate performance models.
%A fine granularity, at lease 15 seconds, is also desirable from our experience.

%RP: Chin, I commented the following line becuause it looks too vague. If you want to clarify this, 
%we can restate in a better way. 
%and the uncertainty come from the interaction between software and hardware components.
%We use moving average to smooth the collected performance data for stable prediction model.
%\rp{Chin: Please explain ``moving average of xxx'', ``why 15 seconds is a desirable window''. These aspects need to be mentioned 
%in a clear manner, not in a handwavy way.}
%We use moving average to construct a stable prediction model.
%Based on our data collected from our internal testbed, 
%it is desirable to use a window greater than 15 seconds. 
%From our experience, a window greater than 15 seconds shows stable results.
%One reason is that a storage request can last longer than 10 seconds for accessing large objects (or files).

\subsubsection{Timestamp alignment}
Proper time synchronization among participating servers is essential to correlate data collected from those servers. 
We use NTP for time synchronization.
The average timestamp of all nodes is taken as the basis for time alignment.

\subsubsection{Feature transformation for a distributed storage system}
%
As mentioned earlier, elasticity is an important feature of SDS, since it needs to adjust its size based on storage demand. 
%The built prediction model needs to be able to predict end-to-end performance at different system scales.
Thus our model must accurately predict end-to-end performance at arbitrary deployment scales.
However, the data collected from different scales may have different dimensions. 
For instance, Ceph with 10 Object Storage Servers (OSDs) generates 10 copies of low-level performance metrics, 
while Ceph with 5 OSDs generates fewer data points. This makes it hard to train and build a unified model.
As mentioned in Section~\ref{sec:feaatures_for_distributed_system}, we use \emph{mean, sum, std, and 5\%} statistical 
variables to capture 
different types of workload distribution such as
\emph{hotspot}, \emph{load imbalance}, and \emph{aggregate performance}.

%To achieve the goal, we need to make sure the dimension of data, collecting from different system configurations are comparable.
%As suggested in Section~\ref{sec:feaatures_for_distributed_system}, our feature transformation describes a distributed system as in one of the four conditions: load-balancing, non-load-balancing, hotspot and the aggregate workload.
%As suggested in Section~\ref{sec:feaatures_for_distributed_system}, our feature transformation process 
%takes four conditions into account: load-balancing, non-load-balancing, hotspot and the aggregate workload.

In summary, Inside-Out collects 32 raw low-level system metrics with one-second granularity.
Inside-Out applies proper time alignment and moving average with one-minute windows for stabilizing performance data.
Then it calculates \emph{mean}, \emph{std}, \emph{sum} and \emph{5\%} of individual metrics collected from multiple machines.
This ensures that our performance model can accept input data for systems with varying scales of deployment, 
while preserving important characteristics of a distributed storage system.
For training and validation purposes, 
we measure end-to-end performance metrics (IOPS, throughput, and latency) every 5 seconds using COSbench \cite{cosbench}, and take average over the one-minute window.
Next, we describe how we build end-to-end performance models in order to capture the relationship between low-level system metrics and end-to-end throughput and IOPS.

%Lastly, we transform low-level performance metrics to create comparable dataset while preserving characteristics 
%of a distributed storage system, using the feature transformation specified in Figure~\ref{fig:feature_types}.

%In summary, we use a 60-seconds window to smooth performance data, and proper time alignment has applied to the monitored performance data.
%\mra{Chin, what did you use to transform the data? average? It would be good if we can briefly state how we transform the data.
%if you already described this somewhere later in the paper, we could forward reference the section.}


\subsection{Exploring Learning Methods}
\label{sec:algorithm_selection}

%We collect time series data composed of low-level system metrics collected from all machines running the distributed storage system. 
%At each sampling time (every $t$ seconds), a low-level performance snapshot of a running distributed storage system is \rp{obtained from} the dataset.
%Given a performance snapshot at time $t$ of a running distributed storage, we aim to build a model that predicts the storage's end-to-end performance, e.g. throughput and IOPS.
%The performance snapshot is the low-level performance metrics collected from distributed storage nodes.

Our goal is to build a model that accurately predicts end-to-end throughput and IOPS by analyzing only the low-level metrics of a distributed storage system.
We explore several algorithms, including statistical regression \cite{Fron2004, hastie2005}, 
decision tree learning and random forests learning \cite{hastie2005,Wang2004}.
For statistical regression, we mainly focus on linear regression techniques, 
which can be extended to support non-linear regression by expanding features that simulate, 
for example, quadratic terms \cite{Kundu2010}.
We did not find this necessary in our application and exclude the discussion.

%\subsubsection{Lasso}
%\textit{Lasso\footnote{Least Absolute Shrinkage and Selection Operator}} 
% remove the full name above
\textit{Lasso} 
is a least square linear regression technique with L1-norm regularization.
The L1 penalty function leads to a sparse solution, which has an effect of restricting 
the number of selected variables.
This property is useful for figuring out important features, especially 
when the number of variables or features is large.
%
%\subsubsection{Ridge}
\textit{Ridge} is similar to Lasso but instead uses L2-norm regularization, 
which has the effect of group selection of variables.
This property does not restrict the number of variables selected by the prediction model 
and therefore, the prediction accuracy might degrade and become inconsistent 
%when the feature space is large.
when the number of input features to the training model is large. 
%when we use a small number of features out of much larger feature space.
%\mra{Chin, I modified the previous sentence. Check if it is ok with you.}
%
%\subsubsection{Elastic Net}
\textit{Elastic Net} combines both advantages---it does group selection while enforcing sparsity.
Based on our data set, Lasso and Elastic Net have similar prediction performance, and Ridge shows larger variance.
%
The \textit{Decision Tree} (DT) learning uses a top-down approach and recursively partitions data to fit target values.
The tree-based model is easy to interpret and scales well to large datasets.
\textit{Random Forests} (RF) is an ensemble method that uses multiple decision trees~\cite{hastie2005}.
%In practice, it is accurate, efficient, and more robust compared to a single decision tree.
RF improves a single decision tree in many ways, e.g., accuracy, efficiency, and robustness.
%Due to space limit, please refer to \cite{hastie2005} for detailed description.

To summarize, linear regression models assume a linear relationship and might oversimplify the storage behavior. 
Nonetheless, it has the potential to exhibit better generalization for extrapolating performance prediction 
for the unknown behavior case (the pattern not included in the training dataset).
On the other hand, the tree-based learning can achieve good model accuracy (perfectly fits the training data), 
but it can easily lead to overfitting problems.
Its prediction accuracy decreases, for example,
under different storage workloads,
as shown in \myfigure{\ref{fig:challenge_generalization}}.


\subsection{Two-level Training}
\label{sec:auto_feature_selection}

%As discussed in Section~\ref{sec:non-deterministic}, 
The fundamental challenge 
in building an effective prediction model from a large set of features is the overfitting problem.
One way to address this problem is to perform manual feature selection.
However, this approach is problematic because the right set of features depend on application types, 
deployment topology, resource constraint, etc.

Instead, we propose a two-level training process that filters out irrelevant features in the first step 
and then builds models by using the reduced set of features in the second step.
To this end, Inside-Out pipelines Ridge and Lasso together, where Ridge filters features in 
coarse-granularity and then Lasso builds the prediction model.
We choose Ridge as the filtering algorithm because it is not a sparse solution and considers all features. 
We then apply exhaustive grid search to find the optimized score for important features.
We use $\alpha \times$ \textit{median(coefficients)} derived from Ridge as the threshold.

For comparison, we consider Decision Tree with Lasso (Auto-DTL) and RandomForest with Lasso (Auto-RFL).
Our evaluation shows Inside-Out outperforms consistently across all prediction cases, 
and boosts prediction accuracy in several scenarios, where the linear regression models 
fail to generalize the behavior of a distributed storage system.
We also experimented by using Lasso and Elastic Net as the filter algorithm but did not find comparable performance with Inside-Out.
%Beside, we also evaluated Lasso and used Elastic Net as a filter algorithm and 
%found they are not comparable with Auto-RL.
%\mra{this sentence is a little confusing..}
%We will leave them as future work to further study the major difference.

Inside-Out uses the following pseudo code to generate an end-to-end performance model.
In practice, we set $k$ to $10$ for stable prediction results.
The data processing part is explained in previous sections.
Features are automatically selected using the Ridge algorithm
with multiple thresholds.
We vary $\alpha$ from $0.1, 0.2, ..., 1.0$.
The grid search approach is used to select the best model.


 \begin{algorithm}
 \caption{Inside-Out Model Building}
 \begin{algorithmic}[1]
 \renewcommand{\algorithmicrequire}{\textbf{Input:}}
 \renewcommand{\algorithmicensure}{\textbf{Output:}}
 \REQUIRE low-level performance metrics from distributed nodes
 \ENSURE  an end-to-end performance model
 \\ \textit{Initialisation}
  \STATE $thresholds = \left \{ \alpha_{1}, ... , \alpha_{N} \right \} $\
  \STATE $m1 =$ filtering algorithm $\rightarrow Ridge$
  \STATE $m2 =$ model algorithm $\rightarrow Lasso$
  \STATE $k =$ k-fold cross validation
  \STATE $score = 0$
 \\ \textit{Data preprocessing} (refer to Section~\ref{sec:data_preprocessing})
  \STATE alignment of input data
  \STATE calculate moving average across metrics
  \STATE feature transformation for the distributed scenario
 \\ \textit{Grid Search}
  \FORALL{$t \in thresholds$}
  \STATE $features =$ execute $m1$ with threshold $t$
  \STATE $score, m = $max(crossvalidation($k$, $m2$, $features$))
  \ENDFOR
 \RETURN $m$ with maximum $score$
 \end{algorithmic} 
 \end{algorithm}
