% Please add the following required packages to your document preamble:
% \usepackage[table,xcdraw]{xcolor}
% If you use beamer only pass "xcolor=table" option, i.e. \documentclass[xcolor=table]{beamer}
\begin{figure*}[!t]
\centering
\begin{tabular}{|l|l|l|l|l|l|l|}
\hline
\thead{Dataset} & \thead{Mean Fault \\ Rate(Guo)} & \thead{Standard \\ Deviation(Guo)}   & \thead{Measurement \\ (Guo)} & \thead{Mean Fault \\ Rate(Us)} & \thead{Standard \\ Deviation(Us)} & \thead{Measurement \\ (Us)} \\ \hline
\textbf{Apache}  & 9.7                           & \cellcolor[HTML]{C0C0C0}10.8 & \cellcolor[HTML]{C0C0C0}29  & 9.9                          & 2.4                         & 16                         \\ \hline
\textbf{LLVM}    & 3.3                           & \cellcolor[HTML]{C0C0C0}2.4  & \cellcolor[HTML]{C0C0C0}64  & 3.3                          & 0.3                         & 32                         \\ \hline
\textbf{X264}    & 6.4                           & \cellcolor[HTML]{C0C0C0}5.7  & \cellcolor[HTML]{C0C0C0}81  & 6.6                          & 0.5                         & 32                         \\ \hline
\textbf{BDBC}    & 7.8                           & \cellcolor[HTML]{C0C0C0}13.2 & \cellcolor[HTML]{C0C0C0}139 & 9.3                          & 6.8                         & 64                         \\ \hline
\textbf{BDBJ}   * & 2.7                           & \cellcolor[HTML]{C0C0C0}2.5  & 48                          & 2.7                          & 0.7                         & 64                         \\ \hline
\textbf{SQLite}  & \cellcolor[HTML]{C0C0C0}7.2   & \cellcolor[HTML]{C0C0C0}4.2  & \cellcolor[HTML]{C0C0C0}566 & 5.6                          & 0.2                         & 64                         \\ \hline
\end{tabular}
\caption{Comparison of
Guo et.al \cite{guo2013variability} with PW number of configurations with WHAT+$S_1$ 
(shown in right-hand columns). Here N refers to the number of features of the software system (please refer to \cite{guo2013variability} for more details). Gray denotes the cases where our method results in lower median fault rate and is more stable i.e. lower standard deviation. We see that our method does better in SQLite and close to prior works results using far less evaluation (except for BDBC).}\label{fig:guo_pw}
\end{figure*}