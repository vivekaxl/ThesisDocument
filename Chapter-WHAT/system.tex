\section{To Eliminate Sub-Optimal Choices}
\label{sec:system}

While the exemplar configuration is adequate for most workloads and reduces measurement cost significantly, it is almost inevitably that fewer workloads may suffer from sub-optimal performance (since we trade-off near-optimal performance for a large reduction in measurement cost).
For instance, 30\% workloads (\textit{c4.large}) underperform ($> 1.4$) as shown in \mytable{\ref{table:top3}}.
Similarly, the 90 percentile in \myfigure{\ref{fig:s2_cost_performance}}.
Although we have shown \micky is much more practical in the knee point analysis (Section~\ref{sec:kneepoint}),
it would be great if we can inform users of those sub-optimal choices.


We propose a two-level approach that integrates our previously built system, \scout, to detect this problem for further optimization~\cite{Hsu2018Scout}.
\scout is able to answer ``is there a better configuration than the current choice?''.
\myfigure{\ref{fig:system_design}} illustrates the proposed system integration.
Users get choices of optimizing those under-performed workloads.
\myfigure{\ref{fig:detection_misprediction}} indicates that those sub-optimal choices are very likely to be identified.
The detection module can detect bad performance with a median accuracy of 98\%.
This is promising because users benefit from
low measurement cost (by \micky) and
performance guarantee (by \scout).
This ability enables users to further optimize for those sub-optimal workloads, which is particularly beneficial to highly recurring workloads.