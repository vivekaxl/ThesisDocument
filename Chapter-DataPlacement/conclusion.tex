\section{Conclusion}
\label{sec:conclusion}

Efficient deployment of large-scale, distributed systems with
an irregular workload requires both cluster sizing and data placement.
We show the uniform distribution is (unsurprisingly) poor for
typical, non-uniform workloads.
This work further shows that coarse-grain replication can reduce
over-utilization but is unable to address under-utilization.
Finer partition granularity reduces both under- and
over-utilization.
With fine-grain partitioning there is a placement decision.
Maximizing the number of unique partitions per node increases robustness
to workload misprediction,
while minimizing the number reduces storage
footprint.
Our empirical study using an HPCC Roxie cluster shows
the benefit of footprint reduction does not offset cost due to poorer
load balancing.
However, we do not believe this is universally true.

This work focuses on the dimension and tradeoffs of various data
replication and placement strategies.
For our future work, we plan to implement an elastic controller
that incorporates our proposed data placement schemes.
In an elastic system, the gains of the optimal placement must be
offset by the cost of data movement.
Calculating this data movement cost remains as a future work.
