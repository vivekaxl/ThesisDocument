\begin{table*}[h]
\caption{Comparisons of Hadoop models with different storage architecture}
\begin{center}
\resizebox{\textwidth}{!}{\begin{tabular}{| l | p{4.5cm} | p{4.5cm} | p{4.5cm} |}
\hline
 & \bf{Hadoop Reference Model} & \bf{Remote Storage Model} & \bf{Cloud Storage Model} \\
 \hline
Resource sharing & Dedicated Hadoop cluster & Enable time and space sharing & Same as the remote storage model \\
\hline
Deployment flexibility & The ratio between computation and storage resources remains fixed & Highly flexible to determine the size of computation and storage resource & Same as the remote storage model \\
\hline
Performance scalability & Scale-out easily & Proportional to storage I/O capability and network bandwidth & Network limitation \\
\hline
Read/Write Performance & Depends on local disk configuration & Support concurrent read/write (high I/O performance) & Long latency for REST and SOAP protocol \\
\hline
Replication & Triple replication is common & Block-level replication & Geo-replication is considered to reduce latency \\
\hline
Hadoop compatibility & Native support & Mounted as local storage, e.g. NFS & Only Amazon S3 is supported by default \\
\hline
Hadoop optimization & Schedulers are optimized for data locality & Not optimized & Not optimized \\
\hline
Extra & Hadoop is optimized for this model, but fails to apply to many use cases & Enterprise storage supports rich functions, e.g. de-duplication, archiving, data lifecycle management & Best fits the cloud computing scenario \\
\hline
\end{tabular}}
\end{center}
\label{tab:model_comparison}
\end{table*}
