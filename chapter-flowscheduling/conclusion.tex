\section{Conclusion}
\label{sec:conclusion}

The decoupled Hadoop model is flexible and much more preferable in many scenarios.
However, existing Hadoop schedulers do not consider this model and hence the scheduling method fails to optimize the system throughput.
Our flow scheduling method uses the penalty cost for task assignments in order to increase the processing flow rate on computing facilities.
We encode this problem as the min-cost flow problem and then we can obtain the optimal assignment.
We have implemented a pluggable Flow Scheduler for Hadoop YARN and it supports the latest version of Hadoop.
Our experiment results have shown that our flow scheduling can greatly improve the system throughput by about 30\% so as to eliminate stragglers.
These results support that the proposed flow scheduling can maintain the flow rate of processing.

Flow scheduling seems efficient for the decoupled model, but there still remains large space to improve.
For our current implementation, Flow Scheduler requires job profile and machine profile, which is not practical.
We believe we can estimate the flow demand of tasks and the flow capability of facilities at runtime.
A naive approach is to sample the flow demand of a task and then use this information to decide the cost of the remaining tasks of the same job.
Another approach is to monitor the flow rate of tasks so that we can adjust the penalty cost dynamically.
We can also decide the flow capability of facilities in a similar way.
Overall, we are positive about flow scheduling but more extensive evaluations have to be conducted before we can conclude.