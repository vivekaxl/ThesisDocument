\section{Introduction}
\label{sec:cat::introduction}

\iffalse
Cloud computing is a cost-effective
alternative to on-premise computing.
Cloud computing (specifically, Infrastructure as a Service) provides a large variety of architectural configurations,
such as the number of cores, amount of memory, and the number of nodes.
Cloud service providers
(such as Amazon, Google, and Azure) offer over 100 virtual machines (VM) types~\cite{Yadwadkar2017}.
The ready flexibility in cloud offerings has created a paradigm shift.
Whereas before a workload was tuned for the cluster that was available, in the cloud
the \emph{architectural configuration can be tuned for the workload}.
This flexibility imposes a burden on users who must choose the \emph{right} architectural configurations.
The wrong choice can increase elapsed time by 20 times and 
cost 10 times compared to the optimal~\cite{Hsu2018Arrow,Hsu2018Scout}.
Users need an effective way to find the right architectural configurations
for their cloud applications.
Therefore,
choosing the right VM type for a workload is essential to provide quality service while being cost effective~\cite{Frey2013, Yao2017}.
\fi


Cloud computing is a cost-effective alternative to on-premise computing.
Choosing the right VM type for a workload is essential to provide quality service
while being cost effective~\cite{Frey2013, Yao2017}.
In this chapter, we describe the \emph{cloud architecture tuning (CAT)} problem
~\cite{Venkataraman2016,Alipourfard2017,Yadwadkar2017,Hsu2018Arrow}.
Given a workload and a service objective, we focus on delivering
the best architecture configurations, such as
virtual machine (VM) types and the number of VMs.
CAT is similar to hyper-parameter tuning in many aspects
~\cite{herodotou2011starfish,zhu2017bestconfig,Dewancker2015,shahriari2016taking,Klein2017,golovin2017google}.
However, the search space of CAT can be more irregular,
which makes existing approach to CAT more fragile~\cite{Hsu2018Arrow}.
This is because the same workload can perform very differently
on two similar configurations.
For example, memory bottleneck can exist in one configuration but not another.

A good solution for CAT should have 
low search cost and find (near) optimal configurations.
Additionally, the solution should be reliable, scalable, and work for wide range of applications and workloads.
\emph{Ernest} is an effective method to extrapolate workload performance
on different configurations, but it is not scalable because
it requires separate prediction models
for distinct workloads and VM types~\cite{Venkataraman2016}.
\emph{CherryPick} adopts Bayesian Optimization to support
any kinds of workloads~\cite{Alipourfard2017}.
However, it relies on an appropriate kernel function to model the search space,
which makes it fragile~\cite{Hsu2018Arrow}.
\emph{PARIS} uses extensive training data
to build prediction model~\cite{Yadwadkar2017}.
However, it may suffer from a high variance of prediction error.
Thus, no prior work fulfills all the requirements of a good solution.