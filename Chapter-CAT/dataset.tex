\section{Open Performance Data}
\label{sec:cat::dataset}

To evaluate a \cat method,
we need large-scale performance dataset---evaluating diverse workloads on different architectural choices.
We conducted a large-scale evaluation using different workloads and software systems on Amazon Web Services (AWS)~\cite{aws}.
We choose Apache Hadoop (version 2.7) and Apache Spark (version 1.5 and 2.1) as our software system~\cite{hadoop, spark}.
Our evaluation includes data processing, OLAP queries, and machine learning, which are popular applications on Hadoop and Spark.
We choose 18 VMs and run the 30 applications on them. \mytable{\ref{tab:dataset}} lists all the software systems and
applications.



\begin{table}[!htbp]
\centering

\caption{The evaluated applications. In total, there are 30 applications and 107 workloads measured on Hadoop 2.7, Spark 1.5 and Spark 2.1.}
\label{tab:dataset}
\resizebox{.95\linewidth}{!}{%
\begin{tabular}{@{}p{2.5cm}p{14cm}@{}}
\toprule
\textbf{Application} & \textbf{Description} \\ \midrule
\multicolumn{2}{l}{\noindent{\textbf{Micro Benchmark}}} \\
sort & Sorts text input data, generated by RandomTextWriter in Hadoop. \\
terasort & A standard Hadoop benchmark. Data is generated from TeraGen. \\
pagerank & The PageRank algorithm. Hyperlinks follow the Zipfian distribution. \\
wordcount & Counts the frequency of words that generated by RandomTextWriter.  This is a typical MapReduce job. \\ \midrule
\multicolumn{2}{l}{\textbf{OLAP}} \\
aggregation & A Hive query performing aggregation. \\
join & Implement the join operation in Hive \\
scan & Implement the scan operation in Hive \\ \midrule
\multicolumn{2}{l}{\textbf{Statistics Function}} \\ 
chi-feature & Chi-square Feature Selection. \\
chi-gof & Chi-Square Goodness of Fit Test. \\
chi-mat & Chi-square Tests for identity matrix. \\
spearman & Compute Spearman's Correlation of two RDDs. \\
statistics & Generate column-wise summary statistics. \\
pearson & Compute the Pearson's correlation of two series of data. \\
svd & Singular Value Decomposition, a fundamental matrix operation for finding approximate solutions.\\
pca & Principal Component Analysis for dimension reduction. \\
word2vec & Generate distributed vector presentation of words according to distance. \\ \midrule
\multicolumn{2}{l}{\textbf{Machine Learning}} \\
classification & Implement the generalized linear classification model. \\
regression & Generalized Linear Regression Model. \\
als & The Alternating Least Squares algorithm, implemented in spark.mllib. It is a collaborative filtering algorithm used for product recommendation. \\
bayes & Implements the Naive Bayes algorithm for the multiclass classification problem. Input documents are generated from /usr/share/dict/linux.words.ords. \\
lr & A popular algorithm for the classification problem. \\
mm & Matrix multiplication with configurable row, column and block sizes.\\
d-tree & A greedy algorithm for classification and regression problems. \\
gb-tree & Gradient Boosted Tree, an ensemble learning method for classification and regression problems. \\
df & The Random Forest algorithm for classification and regression problems. \\
fp-growth & The FP-growth algorithm to mine frequent pattern in large-scale dataset. \\
gmm & Gaussian Mixture Model is a clustering algorithm that uses k Gaussian distributions to find the k clusters. \\
kmeans & K-means is a common clustering algorithm that finds k cluster centers. \\
lda & Latent Dirichlet allocation is a clustering algorithm that infers topics from a collection of text documents. \\
pic & Power iteration clustering is a scalable algorithm for clustering. \\ \bottomrule
\end{tabular}
}
\end{table}



We also vary the input size or input parameters to the applications for creating diverse workloads~\cite{Dalibard2017}.
When workloads are different, the optimal VM type (even for the same application) might change as well.
%Consequently, the optimal VM type for a given workload with different inputs might also change.
% Throughout this paper, we use workloads to represent applications with different inputs.
By running workloads with different data sizes, we can observe whether a particular VM can sustain increasing resource requirements (of a workload).
Our motivation (for the large-scale study) was to diversify the workloads such that we can extensively benchmark VMs.
In this study, each workload is tested with three different inputs sizes.
Some tests failed because smaller VM instances run out of memory.
Those are excluded in our data set.
In total, we measure the performance and collect the low-level information of 107 workloads on 18 different VM types.
We also collect 18 workloads on 69 multi-node configurations.
\mytable{\ref{tab:dataset_overview}} summarizes the dataset collection.

\begin{table}[!htbp]
\centering
\caption{\small{\textbf{Dataset description.}
The benchmark programs are taken from \emph{HiBench}~\cite{hibench} and \emph{spark-perf}~\cite{sparkperf}.
}}
%\resizebox{.98\linewidth}{!}{%
\begin{tabular}{ll|ll}
\hline
\textbf{Applications}               & 30      & \textbf{Evaluation}              & $> 12,000$ runs  \\ \hline
\textbf{Workloads}                  & 125     & \textbf{Duration}              & $> 1,300$ hours  \\ \hline
\textbf{Single-node}     & 18        & \textbf{Raw data size}              & 2.5 GB \\ \hline
\textbf{Multi-node} & 68        & \textbf{Data points}                & 6 million \\ \hline
\textbf{Frameworks}  &   \begin{tabular}{@{}l@{}}Hadoop \\ Spark\end{tabular} & \textbf{Low-Level Metrics} &  \begin{tabular}{@{}l@{}} 72 (raw) \\ 504 (populated) \end{tabular} \\ \hline
\end{tabular}
%}
\label{tab:dataset_overview}
\end{table}


Performance optimization requires continuous efforts
to keep up with the rapid pace of cloud computing.
In our experience, performance data is hard to find.
A lack of performance data discourages the advances
in system performance research.
We believe that we will see advances in performance optimization
by sharing performance data.
Our large-scale performance dataset is available at
\url{https://github.com/oxhead/scout}.
